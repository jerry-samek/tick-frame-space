\section{Law-000: XOR Parity Rule}

\subsection{Status}
Draft 1

\subsection{Purpose}
Define the minimal, parameter-free evolution rule for a discrete substrate represented as a graph without coordinates, geometry, or physical constants.

\subsection{1. Substrate: Coordinate-Free Graph}

The substrate consists of:

\begin{itemize}
    \item A set of nodes \(N\), finite or countably infinite.
    \item A set of edges \(E \subseteq N \times N\), representing adjacency only.
    \item Binary node states:
    \[
        S_t : N \rightarrow \{0,1\}
    \]
    where \(S_t(n)\) is the state of node \(n\) at tick \(t\).
    \item A global discrete tick index \(t \in \mathbb{Z}_{\ge 0}\).
\end{itemize}

Edges are assumed symmetric:
\[
    (n,m) \in E \Rightarrow (m,n) \in E.
\]

The substrate contains no coordinates, distances, metrics, or physical parameters.

\subsection{2. Local Quantities}

For a node \(n \in N\):

\paragraph{Neighbors}
\[
    \mathrm{Nbr}(n) = \{ m \in N \mid (n,m) \in E \}.
\]

\paragraph{Active Neighbor Count}
\[
    K_t(n) = \sum_{m \in \mathrm{Nbr}(n)} S_t(m).
\]

\paragraph{Parity}
\[
    P_t(n) = K_t(n) \bmod 2.
\]

\subsection{3. Evolution Rule (Law-000)}

The state update rule is:

\[
    S_{t+1}(n) = S_t(n) \oplus P_t(n),
\]

where \(\oplus\) denotes logical XOR.

\subsubsection{Verbal Description}

\begin{itemize}
    \item Count active neighbors of node \(n\).
    \item Compute parity (even/odd).
    \item If parity is even, the node keeps its state.
    \item If parity is odd, the node flips its state.
\end{itemize}

\subsubsection{Properties}

\begin{itemize}
    \item Binary
    \item Local
    \item Deterministic
    \item Parameter-free
    \item Coordinate-free
    \item Iterative
\end{itemize}

\subsection{4. Global Evolution}

The rule is applied synchronously:

\[
    S_{t+1}(n) = S_t(n) \oplus \left( \sum_{m \in \mathrm{Nbr}(n)} S_t(m) \bmod 2 \right).
\]

This produces a trajectory:
\[
    S_0 \rightarrow S_1 \rightarrow S_2 \rightarrow \dots
\]

\subsection{5. Worked Examples}

\subsubsection{Example 1: 3-Node Line}

Graph: \(0 - 1 - 2\)

Initial state:
\[
    S_0 = (0,1,0)
\]

Evolution:
\[
    S_1 = (1,1,1)
\]
\[
    S_2 = (0,1,0)
\]

Period: 2.

\subsubsection{Example 2: 4-Node Cycle}

Graph: \(0 - 1 - 2 - 3 - 0\)

Initial:
\[
    S_0 = (1,0,0,0)
\]

Evolution:
\[
    S_1 = (1,1,0,1)
\]
\[
    S_2 = (0,1,1,1)
\]
\[
    S_3 = (0,0,1,0)
\]
\[
    S_4 = (1,0,0,0)
\]

Period: 4.

\subsection{6. Interpretation}

Law-000 defines only the substrate and its evolution.
It does \emph{not} define space, time, geometry, motion, or entities.
These emerge from:

\begin{itemize}
    \item the graph structure,
    \item the iteration of the rule,
    \item the observer's sampling process.
\end{itemize}

%%%%%%%%%%%%%%%%%%%%%%%%%%%%%%%%%%%%%%%%%%%%%%%%%%%%%%%%%%%%%%
\section{Validation Suite for Law-000}

\subsection{Purpose}
Provide tools for evaluating whether Law-000 (or any candidate base law) produces coherent emergent structure.

Includes:

\begin{enumerate}
    \item Search criteria for a base law.
    \item Deterministic test vectors.
    \item Emergent \(\pi\) Probe.
\end{enumerate}

%%%%%%%%%%%%%%%%%%%%%%%%%%%%%%%%%%%%%%%%%%%%%%%%%%%%%%%%%%%%%%
\subsection{1. Search for Base Law}

A candidate rule \(R\) is viable if it satisfies:

\paragraph{Minimality}
Binary, local, parameter-free, coordinate-free.

\paragraph{Determinism}
Same input → same evolution.

\paragraph{Universality}
Same rule for all nodes.

\paragraph{Emergent Richness}
Non-trivial dynamics, oscillations, propagating patterns.

\paragraph{Geometric Coherence}
Stable effective dimension and measurable invariants.

%%%%%%%%%%%%%%%%%%%%%%%%%%%%%%%%%%%%%%%%%%%%%%%%%%%%%%%%%%%%%%
\subsection{2. Test Vectors}

\subsubsection{Test Vector A: 3-Node Line}

\[
    S_0 = (0,1,0)
\]
\[
    S_1 = (1,1,1)
\]
\[
    S_2 = (0,1,0)
\]

\subsubsection{Test Vector B: 4-Node Cycle}

\[
    S_0 = (1,0,0,0)
\]
\[
    S_1 = (1,1,0,1)
\]
\[
    S_2 = (0,1,1,1)
\]
\[
    S_3 = (0,0,1,0)
\]
\[
    S_4 = (1,0,0,0)
\]

\subsubsection{Test Vector C: 5-Node Star}

Initial:
\[
    S_0 = (1,0,0,0,0)
\]

Evolution:
\[
    S_1 = (1,1,1,1,1)
\]
\[
    S_2 = (0,1,1,1,1)
\]
\[
    S_3 = (0,0,0,0,0)
\]

%%%%%%%%%%%%%%%%%%%%%%%%%%%%%%%%%%%%%%%%%%%%%%%%%%%%%%%%%%%%%%
\section{Emergent \texorpdfstring{$\pi$}{pi} Probe}

\subsection{Purpose}
A coordinate-free geometric diagnostic for detecting whether a rule produces coherent effective geometry.

\subsection{1. Definitions}

Graph distance:
\[
    d(n,m) = \text{minimum number of edges in a path from } n \text{ to } m.
\]

Ball:
\[
    B(r) = \{ m \mid d(n,m) \le r \}.
\]

Sphere:
\[
    S(r) = \{ m \mid d(n,m) = r \}.
\]

\subsection{2. Effective Dimension}

Estimate:
\[
    |B(r)| \sim r^{d_{\text{eff}}}
\]

Thus:
\[
    d_{\text{eff}} \approx \frac{\Delta \log |B(r)|}{\Delta \log r}.
\]

\subsection{3. Emergent \texorpdfstring{$\pi$}{pi}}

Define:
\[
    V(r) = |B(r)|, \quad A(r) = |S(r)|.
\]

Estimate:
\[
    \pi_{\text{eff}}(r) = \frac{A(r)}{4 r^2}
\]

or:
\[
    \pi_{\text{eff}}(r) = \frac{3 V(r)}{4 r^3}.
\]

\subsection{4. Validation Criteria}

A rule is geometrically coherent if:

\begin{itemize}
    \item \(d_{\text{eff}}\) stabilizes,
    \item \(\pi_{\text{eff}}\) stabilizes,
    \item variance across sample nodes is low,
    \item variance across ticks is low
