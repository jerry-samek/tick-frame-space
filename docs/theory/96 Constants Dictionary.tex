\documentclass[12pt]{article}
\usepackage{amsmath}
\usepackage{amssymb}
\usepackage{geometry}
\usepackage{setspace}
\geometry{margin=1in}
\setstretch{1.25}

\begin{document}

    \begin{center}
    {\LARGE \textbf{Constants Dictionary}}\\[6pt]
    {\large Mapping simulated constants to physical-like values}
    \end{center}

    \section*{Reference Tick}
    We anchor the mapping at tick $t^\ast = 3760$, where the simulation is cooled and stable.

    At this tick, the simulation produced:
    \[
        \alpha_{\text{sim}} = 0.9678899083,\quad
        G_{\text{sim}} = 1.0,\quad
        h_{\text{drift,sim}} = 246.3095167,\quad
        \Lambda_{\text{drift,sim}} = 1.2001668 \times 10^{-5}.
    \]

    We map these to real-world constants:
    \[
        \alpha_{\text{real}} \approx 7.29735257 \times 10^{-3},
    \]
    \[
        G_{\text{real}} \approx 6.67430 \times 10^{-11},
    \]
    \[
        h_{\text{real}} = 6.62607015 \times 10^{-34},
    \]
    \[
        \Lambda_{\text{real}} \approx 1.1 \times 10^{-52}.
    \]

    \section*{Scaling Factors}
    We define linear scaling factors:
    \[
        s_\alpha = \frac{\alpha_{\text{real}}}{\alpha_{\text{sim}}(t^\ast)},
        \quad
        s_G = \frac{G_{\text{real}}}{G_{\text{sim}}(t^\ast)},
    \]
    \[
        s_h = \frac{h_{\text{real}}}{h_{\text{drift,sim}}(t^\ast)},
        \quad
        s_\Lambda = \frac{\Lambda_{\text{real}}}{\Lambda_{\text{drift,sim}}(t^\ast)}.
    \]

    Numerically:
    \[
        s_\alpha \approx 7.5 \times 10^{-3},\quad
        s_G \approx 6.67430 \times 10^{-11},
    \]
    \[
        s_h \approx 2.7 \times 10^{-36},\quad
        s_\Lambda \approx 9.2 \times 10^{-48}.
    \]

    \section*{Mapping Formulas}
    For any tick $t$, the UI computes:
    \[
        \alpha_{\text{phys}}(t) = s_\alpha \cdot \alpha_{\text{sim}}(t),
    \]
    \[
        G_{\text{phys}}(t) = s_G \cdot G_{\text{sim}}(t),
    \]
    \[
        h_{\text{phys}}(t) = s_h \cdot h_{\text{drift,sim}}(t),
    \]
    \[
        \Lambda_{\text{phys}}(t) = s_\Lambda \cdot \Lambda_{\text{drift,sim}}(t).
    \]

    These produce physical-like values that:
    \begin{itemize}
        \item match real constants at the reference tick,
        \item drift realistically at earlier ticks,
        \item stabilize at later ticks,
        \item remain intuitive for users.
    \end{itemize}

    \section*{Angular Pipe (UI Layer)}
    A simple UI mapping can be implemented as:
    \[
        \text{value}_{\text{phys}} = s_{\text{type}} \cdot \text{value}_{\text{sim}}.
    \]

    \section*{Notes}
    \begin{itemize}
        \item This mapping does not claim the simulation reproduces real physics.
        \item It provides a human-friendly interpretation layer.
        \item It allows the UI to show constants in familiar units.
        \item The reference tick can be changed later if needed.
    \end{itemize}

\end{document}
